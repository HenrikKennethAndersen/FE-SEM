% interactcadsample.tex
% v1.03 - April 2017

\documentclass[]{interact}

\usepackage{epstopdf}% To incorporate .eps illustrations using PDFLaTeX, etc.
\usepackage{subfigure}% Support for small, `sub' figures and tables
%\usepackage[nolists,tablesfirst]{endfloat}% To `separate' figures and tables from text if required

\usepackage{natbib}% Citation support using natbib.sty
\bibpunct[, ]{(}{)}{;}{a}{}{,}% Citation support using natbib.sty
\renewcommand\bibfont{\fontsize{10}{12}\selectfont}% Bibliography support using natbib.sty

\theoremstyle{plain}% Theorem-like structures provided by amsthm.sty
\newtheorem{theorem}{Theorem}[section]
\newtheorem{lemma}[theorem]{Lemma}
\newtheorem{corollary}[theorem]{Corollary}
\newtheorem{proposition}[theorem]{Proposition}

\theoremstyle{definition}
\newtheorem{definition}[theorem]{Definition}
\newtheorem{example}[theorem]{Example}

\theoremstyle{remark}
\newtheorem{remark}{Remark}
\newtheorem{notation}{Notation}

% see https://stackoverflow.com/a/47122900
\usepackage{color}
\usepackage{fancyvrb}
\newcommand{\VerbBar}{|}
\newcommand{\VERB}{\Verb[commandchars=\\\{\}]}
\DefineVerbatimEnvironment{Highlighting}{Verbatim}{commandchars=\\\{\}}
% Add ',fontsize=\small' for more characters per line
\usepackage{framed}
\definecolor{shadecolor}{RGB}{248,248,248}
\newenvironment{Shaded}{\begin{snugshade}}{\end{snugshade}}
\newcommand{\AlertTok}[1]{\textcolor[rgb]{0.94,0.16,0.16}{#1}}
\newcommand{\AnnotationTok}[1]{\textcolor[rgb]{0.56,0.35,0.01}{\textbf{\textit{#1}}}}
\newcommand{\AttributeTok}[1]{\textcolor[rgb]{0.77,0.63,0.00}{#1}}
\newcommand{\BaseNTok}[1]{\textcolor[rgb]{0.00,0.00,0.81}{#1}}
\newcommand{\BuiltInTok}[1]{#1}
\newcommand{\CharTok}[1]{\textcolor[rgb]{0.31,0.60,0.02}{#1}}
\newcommand{\CommentTok}[1]{\textcolor[rgb]{0.56,0.35,0.01}{\textit{#1}}}
\newcommand{\CommentVarTok}[1]{\textcolor[rgb]{0.56,0.35,0.01}{\textbf{\textit{#1}}}}
\newcommand{\ConstantTok}[1]{\textcolor[rgb]{0.00,0.00,0.00}{#1}}
\newcommand{\ControlFlowTok}[1]{\textcolor[rgb]{0.13,0.29,0.53}{\textbf{#1}}}
\newcommand{\DataTypeTok}[1]{\textcolor[rgb]{0.13,0.29,0.53}{#1}}
\newcommand{\DecValTok}[1]{\textcolor[rgb]{0.00,0.00,0.81}{#1}}
\newcommand{\DocumentationTok}[1]{\textcolor[rgb]{0.56,0.35,0.01}{\textbf{\textit{#1}}}}
\newcommand{\ErrorTok}[1]{\textcolor[rgb]{0.64,0.00,0.00}{\textbf{#1}}}
\newcommand{\ExtensionTok}[1]{#1}
\newcommand{\FloatTok}[1]{\textcolor[rgb]{0.00,0.00,0.81}{#1}}
\newcommand{\FunctionTok}[1]{\textcolor[rgb]{0.00,0.00,0.00}{#1}}
\newcommand{\ImportTok}[1]{#1}
\newcommand{\InformationTok}[1]{\textcolor[rgb]{0.56,0.35,0.01}{\textbf{\textit{#1}}}}
\newcommand{\KeywordTok}[1]{\textcolor[rgb]{0.13,0.29,0.53}{\textbf{#1}}}
\newcommand{\NormalTok}[1]{#1}
\newcommand{\OperatorTok}[1]{\textcolor[rgb]{0.81,0.36,0.00}{\textbf{#1}}}
\newcommand{\OtherTok}[1]{\textcolor[rgb]{0.56,0.35,0.01}{#1}}
\newcommand{\PreprocessorTok}[1]{\textcolor[rgb]{0.56,0.35,0.01}{\textit{#1}}}
\newcommand{\RegionMarkerTok}[1]{#1}
\newcommand{\SpecialCharTok}[1]{\textcolor[rgb]{0.00,0.00,0.00}{#1}}
\newcommand{\SpecialStringTok}[1]{\textcolor[rgb]{0.31,0.60,0.02}{#1}}
\newcommand{\StringTok}[1]{\textcolor[rgb]{0.31,0.60,0.02}{#1}}
\newcommand{\VariableTok}[1]{\textcolor[rgb]{0.00,0.00,0.00}{#1}}
\newcommand{\VerbatimStringTok}[1]{\textcolor[rgb]{0.31,0.60,0.02}{#1}}
\newcommand{\WarningTok}[1]{\textcolor[rgb]{0.56,0.35,0.01}{\textbf{\textit{#1}}}}

% Pandoc citation processing

\usepackage{hyperref}
\usepackage[utf8]{inputenc}
\usepackage{booktabs}
\usepackage{bm}
\usepackage{mathtools}
\usepackage{amssymb}
\usepackage{amsmath}
\usepackage{tikz}
\usetikzlibrary{arrows}
\usepackage[nofiglist]{endfloat}
\usepackage{blkarray}
\usepackage{setspace}
\usepackage{etoolbox}
\DeclareMathOperator{\E}{\mathbb{E}}
\DeclareMathOperator{\Var}{\mathrm{Var}}
\DeclareMathOperator{\Cov}{\mathrm{Cov}}
\DeclareMathOperator*{\argmax}{arg\,max}
\DeclareMathOperator*{\argmin}{arg\,min}
\mathtoolsset{showonlyrefs}
\BeforeBeginEnvironment{equation}{\begin{singlespace}\vspace*{-\baselineskip}}
\AfterEndEnvironment{equation}{\end{singlespace}\noindent\ignorespaces}
\BeforeBeginEnvironment{align}{\begin{singlespace}\vspace*{-\baselineskip}}
\AfterEndEnvironment{align}{\end{singlespace}\noindent\ignorespaces}
\def\tightlist{}
\pagenumbering{gobble}


\begin{document}

\articletype{}

\title{Supplementary materials for:\\
A closer look at fixed effects regression in structural equation
modeling using \texttt{lavaan}}


\author{\name{Henrik Kenneth Andersen$^{a}$}
\affil{$^{a}$Chemnitz University of Technology, Institute of Sociology,
Chair for Empirical Social Research, Thüringer Weg 9, 09126 Chemnitz,
Germany}
}

\thanks{CONTACT Henrik Kenneth
Andersen. Email: \href{mailto:henrik.andersen@soziologie.tu-chemnitz.de}{\nolinkurl{henrik.andersen@soziologie.tu-chemnitz.de}}}

\maketitle



\singlespacing

\doublespacing

\begin{center}\rule{0.5\linewidth}{0.5pt}\end{center}

The following goes into some more detail on the comparison of
traditional FE and FE-SEM models and discusses several opportunities to
extend the basic FE model outlined in the main article, and provides
concrete guidance on the implementation in \texttt{lavaan}. First, I
verify that the FE-SEM model does, in fact, return essentially identical
results compared to the more traditional methods. Then, I go over a
number of possibilities to relax assumptions associated with the
traditional FE model. Then, I discuss the issue of measurement error and
show how we can use latent variables to deal with it and properly
estimate the coefficients of interest. Then, I show a type of hybrid
FE/RE model that allows us to control for time-invariant unobserved
heterogeneity while including time-invariant predictors in the model.

\hypertarget{model-notation}{%
\section{Model notation}\label{model-notation}}

The following is an outline of matrix notation for those that prefer it
to the path diagrams and model syntax. There are a number of different
model notations (see, for example \citet{Bollen1989} for an overview),
but the one that will serve us best is one that was proposed by
\citet{Graff1979}: \begin{align}
\bm{y}^{*} & = \bm{\Lambda_{y}}^{*} \bm{\eta}^{*}, \\
\bm{\eta}^{*} & = \bm{B}\bm{\eta}^{*} + \bm{\zeta}^{*}, 
\end{align} where
\(\bm{\eta}^{*} = (\bm{y}, \bm{x}, \bm{\eta}, \bm{\xi})^{\intercal}\),
\(\bm{\zeta}^{*} = (\bm{\varepsilon}, \bm{\delta}, \bm{\zeta}, \bm{\xi})^{\intercal}\),
\(\bm{y}^{*} = (\bm{y}, \bm{x})^{\intercal}\). \(\bm{y}\) is a vector of
observed dependent variables and \(\bm{x}\) is a vector of observed
independent variables. \(\bm{\eta}\) is a vector of the latent dependent
variables and \(\bm{\xi}\) is a vector of latent independent variables.
\(\bm{\varepsilon}\) and \(\bm{\delta}\) are vectors of the errors of
the observed dependent and independent variables, respectively, and
\(\bm{\zeta}\) is a vector of the errors, or disturbances, of the latent
variables. Notice the \(^{*}\) symbol is just meant to differentiate the
vectors with them from those without them. That means, \(\bm{\eta}^{*}\)
is a vector that holds the observed and latent variables, both dependent
(in SEM they are referred to as `endogenous') and independent (i.e.,
`exogeneous')\footnote{For our purposes, the terms endogenous and
  dependent, on the one hand, and exogenous and independent, on the
  other, can be used interchangeably.}, \(\bm{\zeta}^{*}\) holds the
errors for the observed variables and the disturbances of the latent
variables. \(\bm{y}^{*}\) holds just the observed variables, both
dependent and independent, and \(\bm{\Lambda_{y}}^{*}\) is a matrix of
ones and zeros that selects the observed variables from
\(\bm{\eta}^{*}\). Lastly, \(\bm{B}\) is a matrix that holds the
regression coefficients.\footnote{If we say that \(p\) and \(q\) stand
  for the number of observed dependent and independent variables,
  respectively, and \(m\) and \(n\) stand for the number of latent
  dependent and independent variables, respectively, then
  \(\bm{\eta}^{*}\) and \(\bm{\zeta}^{*}\) are \(p + q + m + n\),
  \(\bm{y}^{*}\) is \(p + q\), \(\bm{\Lambda_{y}}^{*}\) is
  \((p + q) \times (p + q + m + n)\) and \(\bm{B}\) is
  \((p + q + m + n) \times (p + q + m + n)\) \citep{Bollen1989}.}

This notation may be confusing at first, but it has advantages. First,
it allows us the flexibility we need for the models. For example, it
allows observed \(\bm{x}\) to directly influence observed \(\bm{y}\)
(more common notation assumes that substantive effects occur only
between latent variables, observed ones are only used as indicators, see
for example \citet{Bollen1989}, \citet{Kline2016}). It also allows
\(\bm{\xi}\), i.e., any latent exogenous variables, to influence
\(\bm{y}\) directly. These two scenarios cover the traditional FE model
with observed variables, and one in which latent variables are used to
account for measurement error in the independent variables. It is also
consistent with the notation used for these models in \texttt{lavaan}.
In fact, \texttt{lavaan} switches automatically between matrix notations
depending on the specified model. That means the matrix representation
of the model one sees if they type in
\texttt{lavInspect(model,\ what\ =\ "est")} after specifying their model
in \texttt{lavaan} will match up with the notation used here.\footnote{In
  \texttt{Mplus}, the model matrices can be requested by including
  \texttt{OUTPUT:\ TECH1} in the input file.} It does have a potential
disadvantage however. Besides being less intuitive than the typical
\(\bm{\eta} = \bm{B}\bm{\eta} + \bm{\Gamma}\bm{\xi} + \bm{\zeta}\)
notation, it means that by including the observed covariates in the
stacked long vector \(\bm{y}^{*}\), they are treated as another response
variable with variances and covariances to be estimated by the model
(instead of just using the sample statistics). This means that the
assumption of multivariate normality (otherwise just imposed on the
dependent variables) also applies to the independent ones
\citep[p.~75]{Skrondal2004}. This can be problematic for noncontinuous
independent variables like sex, nationality dummies, marriage status
(married/unmarried), etc. See \citet{Skrondal2004} for more on this
topic.

Let us, however, make things more concrete and take a look at a simple,
three-wave version of the typical FE-SEM using this notation (shown
graphically in Figure \ref{fig:fesem}). For that, we have the following
matrix notation (with labels on the outside of the matrices):
\begin{align}
\begin{split}
\bm{y}^{*} & = \bm{\Lambda_{y}}^{*}\bm{\eta}^{*} \\
\begin{pmatrix}
y_{1} \\
y_{2} \\
y_{3} \\
x_{1} \\
x_{2} \\
x_{3}
\end{pmatrix} & = 
\begin{blockarray}{cccccccc}
 & y_{1} & y_{2} & y_{3} & x_{1} & x_{2} & x_{3} & \alpha \\
 \begin{block}{c(ccccccc)}
 y_{1} & 1 & 0 & 0 & 0 & 0 & 0 & 0 \\
 y_{2} & 0 & 1 & 0 & 0 & 0 & 0 & 0 \\
 y_{3} & 0 & 0 & 1 & 0 & 0 & 0 & 0 \\ 
 x_{1} & 0 & 0 & 0 & 1 & 0 & 0 & 0 \\
 x_{2} & 0 & 0 & 0 & 0 & 1 & 0 & 0 \\
 x_{3} & 0 & 0 & 0 & 0 & 0 & 1 & 0 \\
 \end{block}
\end{blockarray}
\begin{pmatrix}
y_{1} \\
y_{2} \\
y_{3} \\
x_{1} \\
x_{2} \\
x_{3} \\
\alpha
\end{pmatrix},
\end{split} \label{eq:y}
\end{align} \begin{align}
\begin{split}
\bm{\eta}^{*} & = \bm{B}\bm{\eta}^{*} + \bm{\zeta}^{*} \\
\begin{pmatrix}
y_{1} \\
y_{2} \\
y_{3} \\
x_{1} \\
x_{2} \\
x_{3} \\
\alpha
\end{pmatrix} & = 
\begin{blockarray}{cccccccc}
 & y_{1} & y_{2} & y_{3} & x_{1} & x_{2} & x_{3} & \alpha \\
 \begin{block}{c(ccccccc)}
 y_{1}  & 0 & 0 & 0 & \beta & 0 & 0 & 1 \\
 y_{2}  & 0 & 0 & 0 & 0 & \beta & 0 & 1\\
 y_{3}  & 0 & 0 & 0 & 0 & 0 & \beta & 1 \\ 
 x_{1}  & 0 & 0 & 0 & 0 & 0 & 0 & 0 \\
 x_{2}  & 0 & 0 & 0 & 0 & 0 & 0 & 0 \\
 x_{3}  & 0 & 0 & 0 & 0 & 0 & 0 & 0 \\
 \alpha & 0 & 0 & 0 & 0 & 0 & 0 & 0 \\
 \end{block}
\end{blockarray}
\begin{pmatrix}
y_{1} \\
y_{2} \\
y_{3} \\
x_{1} \\
x_{2} \\
x_{3} \\
\alpha
\end{pmatrix} + 
\begin{pmatrix}
\varepsilon_{1} \\
\varepsilon_{2} \\
\varepsilon_{3} \\
x_{1} = \delta_{1} \\
x_{2} = \delta_{2} \\
x_{3} = \delta_{3} \\
\alpha = \xi \\
\end{pmatrix}. 
\end{split} \label{eq:eta}
\end{align} Notice in \(\bm{\zeta}^{*}\), for the independent variables
we could either write, for example \(x_{t}\) or \(\delta_{t}\). As
mentioned above, this is due to the model notation treating the
independent variables like dependent variables with
variances/covariances to be estimated. For the sake of simplicity, we
will ignore this subtlety and refer to the observed variable from now
on, keeping in mind that if the multivariate normality assumption holds,
the estimated statistics will likely be sufficiently close to the sample
ones for it to not make much of a difference.

Admittedly, Equations \eqref{eq:y} and \eqref{eq:eta} may not look like
much yet. We can remedy this by first putting the equation for
\(\bm{\eta}^{*}\) in reduced form, i.e., by getting rid of the dependent
variable on the r.h.s.: \begin{align}
\bm{\eta}^{*} & = \bm{B}\bm{\eta}^{*} + \bm{\zeta}^{*} \\
\bm{\eta}^{*} - \bm{B}\bm{\eta}^{*} & = \bm{\zeta}^{*} \\
(\bm{I} - \bm{B})\bm{\eta}^{*} & = \bm{\zeta}^{*} \\
\bm{\eta}^{*} & = (\bm{I} - \bm{B})^{-1}\bm{\zeta}^{*},
\end{align} where \(\bm{I}\) is the identity matrix. By substituting
this back into the equation for the observed variables we get
\(\bm{y}^{*} = \bm{\Lambda_{y}}^{*}[(\bm{I} - \bm{B})^{-1}\bm{\zeta}^{*}]\),
which works out to: \begin{align}
\begin{pmatrix} 
y_{1} \\
y_{2} \\
y_{3} \\
x_{1} \\
x_{2} \\
x_{3} \\
\end{pmatrix} & = 
\begin{pmatrix}
\alpha + \beta x_{1} + \varepsilon_{1} \\
\alpha + \beta x_{2} + \varepsilon_{2} \\
\alpha + \beta x_{3} + \varepsilon_{3} \\
x_{1} \\
x_{2} \\
x_{3}
\end{pmatrix},
\end{align} which is of course exactly what we should expect given
Equation \eqref{eq:fe}\footnote{We can use the \texttt{sympy} package in
  \texttt{python} to verify and show the steps for this and other
  examples, see the
  \href{https://github.com/henrik-andersen/FE-SEM/blob/master/sympy-doublecheck-matrixnotation.py}{supplementary
  materials} for the code.}.

\hypertarget{assumptions}{%
\subsection{Assumptions}\label{assumptions}}

What essentially differentiates an FE from an RE model is our assumption
concerning the relationship between the unobserved individual effects
and the model covariates \citep{Bollen2010}. The FE model assumes that
\(\mathop{\mathrm{\mathbb{E}}}[\alpha x_{t}] \ne 0\). As such, if we
fail to control for the correlation of the covariate and the
time-invariant part of the error, then the coefficient of interest, here
\(\beta\), will be biased. Our assumption regarding whether the
individual effects are correlated with the model covariates occurs in
\(\mathop{\mathrm{\mathbb{E}}}[\bm{\zeta}^{*}\bm{\zeta}^{* \intercal}] = \bm{\Psi}\),
the covariance matrix of the errors \begin{align}
\bm{y}^{*}\bm{y}^{* \intercal} & = \mathop{\mathrm{\mathbb{E}}}[(\bm{\Lambda_{y}}^{*}(\bm{I} - \bm{B})^{-1}\bm{\zeta}^{*})(\bm{\Lambda_{y}}^{*}(\bm{I} - \bm{B})^{-1}\bm{\zeta}^{*})^{\intercal}] \\
 & = \mathop{\mathrm{\mathbb{E}}}[(\bm{\Lambda_{y}}^{*}(\bm{I} - \bm{B})^{-1}\bm{\zeta}^{*})(\bm{\zeta}^{* \intercal}(\bm{I} - \bm{B})^{-1 \intercal}\bm{\Lambda_{y}}^{* \intercal})] \\
 & = \bm{\Lambda_{y}}^{*}(\bm{I} - \bm{B})^{-1} \mathop{\mathrm{\mathbb{E}}}[\bm{\zeta}^{*}\bm{\zeta}^{* \intercal}] (\bm{I} - \bm{B})^{-1 \intercal}\bm{\Lambda_{y}}^{* \intercal} \\
 & = \bm{\Lambda_{y}}^{*}(\bm{I} - \bm{B})^{-1} \bm{\Psi} (\bm{I} - \bm{B})^{-1 \intercal}\bm{\Lambda_{y}}^{* \intercal}.
\end{align} In the case of an FE model, \(\bm{\Psi}\) will reflect our
belief that the individual effects are correlated with the model
covariates, here again for demonstration the three-wave model:
\begin{align}
\bm{\Psi} & = \mathop{\mathrm{\mathbb{E}}}
\begin{blockarray}{cccccccc}
 & \varepsilon_{1} & \varepsilon_{2} & \varepsilon_{3} & x_{1} & x_{2} & x_{3} & \alpha \\
 \begin{block}{c(ccccccc)}
 \varepsilon_{1} & \varepsilon_{1}^{2} &                     &                     &              &              &              & \\
 \varepsilon_{2} & 0                   & \varepsilon_{2}^{2} &                     &              &              &              & \\
 \varepsilon_{3} & 0                   & 0                   & \varepsilon_{3}^{2} &              &              &              & \\
 x_{1}           & 0                   & 0                   & 0                   & x_{1}^{2}    &              &              & \\
 x_{2}           & 0                   & 0                   & 0                   & x_{2}x_{1}   & x_{2}^{2}    &              & \\
 x_{3}           & 0                   & 0                   & 0                   & x_{3}x_{1}   & x_{3}x_{2}   & x_{3}^{2}    & \\
 \alpha          & 0                   & 0                   & 0                   & \alpha x_{1} & \alpha x_{2} & \alpha x_{3} & \alpha^{2} \\
 \end{block}
\end{blockarray}.
\end{align} Knowing this, we can work out the equation for the
coefficient of interest, \(\beta\). For the sake of simplicity, assume
here and throughout mean-centered variables: \begin{align}
\mathop{\mathrm{\mathrm{Cov}}}(y_{t},x_{t}) & = \mathop{\mathrm{\mathbb{E}}}[y_{t}x_{t}] \\
 & = \mathop{\mathrm{\mathbb{E}}}[(\alpha + \beta x_{t} + \varepsilon_{t})x_{t}] \\
 & = \mathop{\mathrm{\mathbb{E}}}[\alpha x_{t} + \beta x_{t}^{2} + \varepsilon_{t}x_{t}] \\
 & = \mathop{\mathrm{\mathrm{Cov}}}(\alpha, x_{t}) + \beta \mathop{\mathrm{\mathrm{Var}}}(x_{t}) \\
\hat{\beta}_{FE-SEM} & = \frac{\mathop{\mathrm{\mathrm{Cov}}}(y_{t},x_{t}) - \mathop{\mathrm{\mathrm{Cov}}}(\alpha, x_{t})}{\mathop{\mathrm{\mathrm{Var}}}(x_{t})}. 
\end{align} This should make intuitive sense. From the observed
covariance between the dependent and the independent variable, we are
partialling out the part that is due to the covariance between the
independent variable and the individual effects per unit, and then
dividing by the variance of the independent variable, as usual. For the
RE model, we assume \(\mathop{\mathrm{\mathbb{E}}}[\alpha x_{t}] = 0\)
and the equation reduces to
\(\hat{\beta}_{RE-SEM} = \mathop{\mathrm{\mathrm{Cov}}}(y_{t},x_{t})/\mathop{\mathrm{\mathrm{Var}}}(x_{t})\).
The rest of the model-implied covariance matrix results from
\(\bm{y}^{*}\bm{y}^{* \intercal}\).

\hypertarget{a-comparison-with-non-sem-methods}{%
\section{A comparison with non-SEM
methods}\label{a-comparison-with-non-sem-methods}}

Just to be sure that the FE-SEM results do, in fact, line up with the
more traditional methods outlined in Section 2 of the main article, we
can use the long-format data (see
\href{https://github.com/henrik-andersen/FE-SEM}{the supplementary
materials at}) to run the typical FE model using the \texttt{plm}
package \citep{R-plm_a}. By default, the \texttt{plm} function assumes
the dataframe is structured so that the first two columns correspond to
the individual and time indices, see the
\href{https://cran.r-project.org/web/packages/plm/plm.pdf}{documentation}
or \citet{R-plm_a}.

\singlespacing

\begin{Shaded}
\begin{Highlighting}[]
\KeywordTok{library}\NormalTok{(plm)}

\CommentTok{\# Run the FE model in plm }
\NormalTok{fe1 \textless{}{-}}\StringTok{ }\KeywordTok{plm}\NormalTok{(y }\OperatorTok{\textasciitilde{}}\StringTok{ }\NormalTok{x, }
           \DataTypeTok{effect =} \StringTok{"individual"}\NormalTok{, }\DataTypeTok{model =} \StringTok{"within"}\NormalTok{, }
           \DataTypeTok{data =}\NormalTok{ df)}
\KeywordTok{summary}\NormalTok{(fe1)}
\end{Highlighting}
\end{Shaded}

\begin{verbatim}
## Oneway (individual) effect Within Model
## 
## Call:
## plm(formula = y ~ x, data = df, effect = "individual", model = "within")
## 
## Balanced Panel: n = 1000, T = 5, N = 5000
## 
## Residuals:
##      Min.   1st Qu.    Median   3rd Qu.      Max. 
## -3.720902 -0.601550 -0.021365  0.600833  3.238716 
## 
## Coefficients:
##   Estimate Std. Error t-value  Pr(>|t|)    
## x 0.293907   0.015635  18.798 < 2.2e-16 ***
## ---
## Signif. codes:  0 '***' 0.001 '**' 0.01 '*' 0.05 '.' 0.1 ' ' 1
## 
## Total Sum of Squares:    4452.6
## Residual Sum of Squares: 4091.1
## R-Squared:      0.081192
## Adj. R-Squared: -0.14857
## F-statistic: 353.377 on 1 and 3999 DF, p-value: < 2.22e-16
\end{verbatim}

\doublespacing

From this, we see that the results are, indeed, essentially identical,
with \(\hat{\beta}_{FE-SEM} =0.294 \ (0.016)\) and
\(\hat{\beta}_{FE} = 0.294 \ (0.016)\).

Other methods of estimating FE models work in the random or mixed
effects model framework. For example, we can include the cluster means
per individual of the time-varying independent variables, here \(x\), in
the equation to achieve within estimates
\citep{Mundlak1978, Chamberlain1980, Wooldridge2002}.

\singlespacing

\begin{Shaded}
\begin{Highlighting}[]
\CommentTok{\# Generate the cluster means for x per id }
\NormalTok{clusterMeanx \textless{}{-}}\StringTok{ }\KeywordTok{aggregate}\NormalTok{(df}\OperatorTok{$}\NormalTok{x, }\DataTypeTok{by =} \KeywordTok{list}\NormalTok{(df}\OperatorTok{$}\NormalTok{id), }\DataTypeTok{FUN =}\NormalTok{ mean)}
\CommentTok{\# Rename the columns }
\KeywordTok{names}\NormalTok{(clusterMeanx) \textless{}{-}}\StringTok{ }\KeywordTok{c}\NormalTok{(}\StringTok{"id"}\NormalTok{, }\StringTok{"xbar"}\NormalTok{)}

\CommentTok{\# Add the cluster means back into df  }
\NormalTok{df \textless{}{-}}\StringTok{ }\KeywordTok{merge}\NormalTok{(df, clusterMeanx, }\DataTypeTok{by =} \StringTok{"id"}\NormalTok{)}
\end{Highlighting}
\end{Shaded}

\doublespacing

Here using the \texttt{plm} function in the \texttt{random} setup:

\singlespacing

\begin{Shaded}
\begin{Highlighting}[]
\NormalTok{fe2 \textless{}{-}}\StringTok{ }\KeywordTok{plm}\NormalTok{(y }\OperatorTok{\textasciitilde{}}\StringTok{ }\NormalTok{x }\OperatorTok{+}\StringTok{ }\NormalTok{xbar, }
           \DataTypeTok{effect =} \StringTok{"individual"}\NormalTok{, }\DataTypeTok{model =} \StringTok{"random"}\NormalTok{,}
           \DataTypeTok{data =}\NormalTok{ df)}
\KeywordTok{summary}\NormalTok{(fe2)}
\end{Highlighting}
\end{Shaded}

\begin{verbatim}
## Oneway (individual) effect Random Effect Model 
##    (Swamy-Arora's transformation)
## 
## Call:
## plm(formula = y ~ x + xbar, data = df, effect = "individual", 
##     model = "random")
## 
## Balanced Panel: n = 1000, T = 5, N = 5000
## 
## Effects:
##                  var std.dev share
## idiosyncratic 1.0230  1.0115 0.862
## individual    0.1637  0.4046 0.138
## theta: 0.2546
## 
## Residuals:
##       Min.    1st Qu.     Median    3rd Qu.       Max. 
## -4.0697324 -0.6786344 -0.0094158  0.6678915  3.5329988 
## 
## Coefficients:
##              Estimate Std. Error z-value Pr(>|z|)    
## (Intercept) -0.022559   0.019192 -1.1754   0.2398    
## x            0.293907   0.015635 18.7983   <2e-16 ***
## xbar         0.757738   0.024015 31.5532   <2e-16 ***
## ---
## Signif. codes:  0 '***' 0.001 '**' 0.01 '*' 0.05 '.' 0.1 ' ' 1
## 
## Total Sum of Squares:    8878.9
## Residual Sum of Squares: 5112.1
## R-Squared:      0.42424
## Adj. R-Squared: 0.42401
## Chisq: 3682.01 on 2 DF, p-value: < 2.22e-16
\end{verbatim}

\doublespacing

And here using the \texttt{lmer} function of the \texttt{lme4} package
\citep{Bates2015} to estimate a mixed model:

\singlespacing

\begin{Shaded}
\begin{Highlighting}[]
\KeywordTok{library}\NormalTok{(lme4)}

\CommentTok{\# Run the mixed model in lmer with the cluster means for x }
\NormalTok{mixed1 \textless{}{-}}\StringTok{ }\KeywordTok{lmer}\NormalTok{(y }\OperatorTok{\textasciitilde{}}\StringTok{ }\NormalTok{x }\OperatorTok{+}\StringTok{ }\NormalTok{xbar }\OperatorTok{+}\StringTok{ }\NormalTok{(}\DecValTok{1} \OperatorTok{|}\StringTok{ }\NormalTok{id), }\DataTypeTok{data =}\NormalTok{ df)}
\KeywordTok{summary}\NormalTok{(mixed1)}
\end{Highlighting}
\end{Shaded}

\begin{verbatim}
## Linear mixed model fit by REML ['lmerMod']
## Formula: y ~ x + xbar + (1 | id)
##    Data: df
## 
## REML criterion at convergence: 14906.7
## 
## Scaled residuals: 
##     Min      1Q  Median      3Q     Max 
## -3.9358 -0.6464 -0.0106  0.6349  3.2002 
## 
## Random effects:
##  Groups   Name        Variance Std.Dev.
##  id       (Intercept) 0.1637   0.4046  
##  Residual             1.0230   1.0115  
## Number of obs: 5000, groups:  id, 1000
## 
## Fixed effects:
##             Estimate Std. Error t value
## (Intercept) -0.02256    0.01919  -1.175
## x            0.29391    0.01563  18.798
## xbar         0.75774    0.02401  31.553
## 
## Correlation of Fixed Effects:
##      (Intr) x     
## x     0.000       
## xbar  0.010 -0.651
\end{verbatim}

\doublespacing

In both cases, the models return the same estimates as the FE and FE-SEM
models. Also, in both the \texttt{random} setup using the \texttt{plm}
function, and the mixed model using the \texttt{lmer} function, we get
estimates of the variance components, \(\hat{\sigma}^{2}_{\alpha}\) and
\(\hat{\sigma}^{2}_{\varepsilon}\):

\singlespacing

\begin{Shaded}
\begin{Highlighting}[]
\CommentTok{\# Print the variance components for the plm model  }
\KeywordTok{print}\NormalTok{(}\KeywordTok{ercomp}\NormalTok{(fe2), }
      \DataTypeTok{digits =} \DecValTok{3}\NormalTok{)}
\end{Highlighting}
\end{Shaded}

\begin{verbatim}
##                 var std.dev share
## idiosyncratic 1.023   1.011  0.86
## individual    0.164   0.405  0.14
## theta: 0.255
\end{verbatim}

\begin{Shaded}
\begin{Highlighting}[]
\CommentTok{\# Print the variance components for the lmer model }
\KeywordTok{print}\NormalTok{(}\KeywordTok{VarCorr}\NormalTok{(mixed1), }
      \DataTypeTok{comp =} \KeywordTok{c}\NormalTok{(}\StringTok{"Variance"}\NormalTok{, }\StringTok{"Std.Dev"}\NormalTok{), }
      \DataTypeTok{digits =} \DecValTok{3}\NormalTok{)}
\end{Highlighting}
\end{Shaded}

\begin{verbatim}
##  Groups   Name        Variance Std.Dev.
##  id       (Intercept) 0.164    0.405   
##  Residual             1.023    1.011
\end{verbatim}

\doublespacing

From this we see both models report the same estimated variance
components, \(\hat{\sigma}^{2}_{\alpha} = 0.164\) and
\(\hat{\sigma}^{2}_{\varepsilon} = 1.023\), telling us that about 13.8\%
of the residual variance is due to the differences between individuals
(shown in the \texttt{share} column of the \texttt{ercomp()} output).
This is what is referred to as the intraclass correlation coefficient,
or ICC \citep{Hox2010}.

\hypertarget{exten}{%
\section{Extensions}\label{exten}}

\hypertarget{relax}{%
\subsection{Relaxing assumptions meant to mimic traditional FE
models}\label{relax}}

There are a number of implicit assumptions attached to the typical FE
model that can be relaxed in SEM. Some of these assumptions have been
discussed already, and a fairly comprehenisve list of assumptions can be
found in \citet{Bollen2010}. Here, I will go over just a few,
concentrating on the implementation in \texttt{lavaan} and the
opportunity to empirically test whether the adjustments are justified or
not.

The assumptions we will discuss here pertain to the time-invariance of
the effects of both the latent individual effects and the observed
covariates, as well as a time-invariant error variance. We can also
empirically test the correlation between the individual effects and the
covariates to see whether a RE model is preferable to the FE model.

For example, we can rewrite the original FE equation as \begin{align}
y_{it} & = \beta_{t}x_{it} + \lambda_{t}\alpha_{i} + \varepsilon_{it}
\end{align} where \(\beta\) becomes \(\beta_{t}\) and the implicit
regression weight of one turns to \(\lambda_{t}\) to highlight the fact
that the effect of \(x\) as well as \(\alpha\) on \(y\) may vary over
time. We can furthermore easily relax the assumption of time-constant
error variance, i.e., \(\sigma^{2}_{\varepsilon_{t}}\). As noted in the
\href{https://github.com/henrik-andersen/FE-SEM/blob/master/article.pdf}{main
article}, the assumption regarding
\(\mathop{\mathrm{\mathbb{E}}}[\alpha x_{t}]\) in \(\bm{\Psi}\)
determines whether we have an FE or RE model. We can set these to zero
and test whether the RE model would be preferable to the FE model. In
general, if the individual effects are truly uncorrelated with the model
covariates, it is advisable to switch to an RE model since because it
uses up less degrees of freedom, it will have smaller standard errors
\citep{Bollen2010}.

In the following \texttt{lavaan} code, we simply remove the factor
loadings of one for the latent individual effect variable which allows
them to be estimated freely at each timepoint. For the effect of the
covariate, we can either delete the constraints \texttt{b} in
\texttt{yt\ \textasciitilde{}\ b*xt} or give each regression a different
label, e.g., \texttt{b1}, \texttt{b2}, \texttt{b3}, etc. Similarly, to
allow the error variance to vary over time, we turn the constraints
\texttt{e} into simple labels, i.e., \texttt{e1}, \texttt{e2},
\texttt{e3}, etc., or again just delete them. In fact, regarding the
error variances, they will be estimated necessarily, and do not need to
be explicitly mentioned in the model syntax at all. Finally, to move
from an FE to an RE model, we could simply constrain the correlations
between the individual effects and the covariates to zero, i.e.,
\texttt{a\ \textasciitilde{}\textasciitilde{}\ 0*x1\ +\ 0*x2\ +\ 0*x3\ +\ 0*x4\ +\ 0*x5}.

\singlespacing

\begin{Shaded}
\begin{Highlighting}[]
\NormalTok{fe\_sem\_fullyrelaxed \textless{}{-}}\StringTok{ \textquotesingle{}}
\StringTok{\# Define individual effects variable }
\StringTok{a =\textasciitilde{} y1 + y2 + y3 + y4 + y5}
\StringTok{\# Regressions, constrain coefficient to be equal over time}
\StringTok{y1 \textasciitilde{} b1*x1}
\StringTok{y2 \textasciitilde{} b2*x2 }
\StringTok{y3 \textasciitilde{} b3*x3}
\StringTok{y4 \textasciitilde{} b4*x4}
\StringTok{y5 \textasciitilde{} b5*x5}
\StringTok{\# Allow unrestricted correlation between eta and covariates}
\StringTok{a \textasciitilde{}\textasciitilde{} x1 + x2 + x3 + x4 + x5}
\StringTok{\# Alternatively: constrain all to 0 for RE model, or}
\StringTok{\# just individual correlations}
\StringTok{\# a \textasciitilde{}\textasciitilde{} 0*x1 + 0*x2 + 0*x3 + 0*x4 + 0*x5}
\StringTok{x1 \textasciitilde{}\textasciitilde{} x2 + x3 + x4 + x5}
\StringTok{x2 \textasciitilde{}\textasciitilde{} x3 + x4 + x5}
\StringTok{x3 \textasciitilde{}\textasciitilde{} x4 + x5}
\StringTok{x4 \textasciitilde{}\textasciitilde{} x5}
\StringTok{\# Constrain residual variances to be equal over time}
\StringTok{y1 \textasciitilde{}\textasciitilde{} e1*y1}
\StringTok{y2 \textasciitilde{}\textasciitilde{} e2*y2}
\StringTok{y3 \textasciitilde{}\textasciitilde{} e3*y3}
\StringTok{y4 \textasciitilde{}\textasciitilde{} e4*y4}
\StringTok{y5 \textasciitilde{}\textasciitilde{} e5*y5}
\StringTok{\textquotesingle{}}
\NormalTok{fe\_sem\_fullyrelaxed.fit \textless{}{-}}\StringTok{ }\KeywordTok{sem}\NormalTok{( }\DataTypeTok{model =}\NormalTok{ fe\_sem\_fullyrelaxed, }
                                \DataTypeTok{data =}\NormalTok{ dfw, }
                                \DataTypeTok{estimator =} \StringTok{"ML"}\NormalTok{)}
\end{Highlighting}
\end{Shaded}

\doublespacing

\singlespacing

\doublespacing

As outlined in \citet{Bollen2010}, the researcher has the opportunity to
test each of the assumptions empirically and decide whether a more
parsimonious, i.e., restrictive model is justifiable. For each
assumption, a likelihood ratio test can be carried out to determine
whether the improvement to model fit resulting from the relaxation of
various assumptions is significant or whether the more parsimonious
model is preferable after all.

If we use the original model \texttt{fe\_sem.fit} (from the
\href{https://github.com/henrik-andersen/FE-SEM/blob/master/article.pdf}{main
article}) as a starting point, the best strategy for testing these
assumptions is to work in a stepwise fashion, relaxing one assumption at
a time. We can begin by first constraining the correlation between
\(\alpha\) and \(x_{t}\) to zero (\texttt{re\_sem}) for an RE model. If
turning from an FE to an RE model does not significantly worsen model
fit, we can go forward with the rest of the steps with the RE model. If,
however, the fit does worsen significantly, it is likely better to stick
with the FE model; moving forward then with it to see if a less
restrictive FE model is preferable. We can perform a likelihood ratio
test in \texttt{R} using the \texttt{anova()} function:

\small
\singlespacing

\begin{Shaded}
\begin{Highlighting}[]
\KeywordTok{anova}\NormalTok{( fe\_sem.fit, re\_sem.fit)}
\end{Highlighting}
\end{Shaded}

\begin{verbatim}
## Chi-Squared Difference Test
## 
##            Df   AIC   BIC   Chisq Chisq diff Df diff Pr(>Chisq)    
## fe_sem.fit 32 30998 31111  30.137                                  
## re_sem.fit 37 31809 31897 850.928     820.79       5  < 2.2e-16 ***
## ---
## Signif. codes:  0 '***' 0.001 '**' 0.01 '*' 0.05 '.' 0.1 ' ' 1
\end{verbatim}

\doublespacing
\normalsize

The table that is generated shows a comparison of the nested models, in
decending order according to degrees of freedom. The RE model does not
estimate the correlations between the individual effects and the
covariates, so it is more parsimonious and thus listed at the bottom.
The \texttt{Chisq} column shows the \(\chi^{2}\) statistic for both
models and the \texttt{Chisq\ diff} column calculates the difference
between the two. Obviously, according to the DGP, the correlation
between the individual effects and \(x_{t}\) is not zero, so fixing
these to zero leads to a substantial amount of misfit. The last column
puts the \(\chi^{2}\) difference in relation to the difference in
degrees of freedom and gives a p-value for the probability that the
difference is solely due to chance. Here, the change in \(\chi^{2}\) is
highly significant, so the FE model should be retained.

After now having established once and for all that FE is our preferred
model, we can begin relaxing the rest of the assumptions. I show the
following merely as a demonstration of the procedure, we know already
from the DGP that the parsimonious model as specified in
\texttt{fe\_sem.fit} is appropriate. We can next allow the error
variances (\texttt{fe\_semb.fit}), the effect of \(x\) on \(y\)
(\texttt{fe\_semc.fit}) and finally the factor loadings of the
individual effects (\texttt{fe\_semd.fit}) all to vary over time.

\singlespacing

\doublespacing

\small
\singlespacing

\begin{Shaded}
\begin{Highlighting}[]
\KeywordTok{anova}\NormalTok{( fe\_sem.fit, fe\_semb.fit, fe\_semc.fit, fe\_semd.fit)}
\end{Highlighting}
\end{Shaded}

\begin{verbatim}
## Chi-Squared Difference Test
## 
##             Df   AIC   BIC  Chisq Chisq diff Df diff Pr(>Chisq)
## fe_semd.fit 20 31017 31189 25.140                              
## fe_semc.fit 24 31010 31162 25.764     0.6249       4     0.9603
## fe_semb.fit 28 31003 31135 26.686     0.9215       4     0.9215
## fe_sem.fit  32 30998 31111 30.137     3.4516       4     0.4853
\end{verbatim}

\doublespacing
\normalsize

Keep in mind that a less parsimonious model (fewer degrees of freedom)
can never fit worse than a more parsimonious one (more degrees of
freedom). I.e., chance variations due to sampling error mean that adding
constraints to a model will tend to always worsen fit, at least
minimally. The question here is whether the improvement to fit by
loosening constraints is meaningful or not. In the table above, we
should not expect any meaningful improvements moving from
\texttt{fe\_sem.fit} to \texttt{fe\_semd.fit}. Here, using simulated
data, we have the luxury of knowing that any significant differences in
\(\chi^{2}\) are due to chance. With real data, it is up to the
researcher to apply their best judgment and decide whether the results
are plausible or not.

\hypertarget{measerr}{%
\subsection{Measurement error}\label{measerr}}

What if the observed variables are not measured perfectly? Then what we
observe, call them \(\tilde{x}_{t}\) and \(\tilde{y}_{t}\) are
composites of the true score we are after, i.e., \(x_{t}\) and
\(y_{t}\), plus an additive measurement error portion: \begin{align}
\tilde{x}_{t} & = x_{t} + \upsilon_{t}, \\
\tilde{y}_{t} & = y_{t} + \nu_{t}.
\end{align} How does this affect our model? Well, first notice that
measurement error in the dependent variable is typically less of a
serious problem than measurement error in the independent variables. Let
us assume again mean-centered variables so that we can ignore the
intercept, and consider the following simple bivariate equation:
\begin{align}
y & = \beta x + \varepsilon
\end{align} if \(y\) is measured imperfectly and what we observe is
\(\tilde{y} = y + \nu\), then we can rewrite the equation as:
\begin{align}
(\tilde{y} - \nu) & = \beta x + \varepsilon \\
\tilde{y} & = \beta x + \varepsilon + \nu.
\end{align} The measurement error in \(y\) just gets added to the
regression error. As long as \(\nu\) is uncorrelated with \(x\), then
the regression coefficient will be unbiased
\citep{Pischke2007, Wooldridge2009}. However, this will increase the
error variance and thus make the estimates less precise.

We will look at the effect of measurement error in the dependent
variable using an example shortly. For now though, let us be safe in the
knowledge that the coefficient of interest is likely unbiased, and
concentrate on the more serious problem of error in the independent
variable.

The intuition behind the problem of measurement error in the independent
variable(s) can be explained as follows. Take
\(\tilde{x} = x + \upsilon\) and substitute this into the equation for
\(y\): \begin{align}
y & = \beta x + \varepsilon \\
 & = \beta(\tilde{x} - \upsilon) + \varepsilon \\
 & = \beta\tilde{x} + (\varepsilon - \beta\upsilon).
\end{align} Since \(\tilde{x}\) is obviously correlated with
\(\upsilon\) (unless the variance of \(\upsilon\) is so small so that
the correlation is essentially negligible), then the composite error in
this regression is also correlated with the independent variable and
thus the estimated coefficient of \(\beta\) will be biased.

\hypertarget{the-consequences-of-measurement-error}{%
\subsubsection{The consequences of measurement
error}\label{the-consequences-of-measurement-error}}

To demonstrate the effect of measurement error on the FE-SEM model, and
then provide a strategy for dealing with measurement error in SEM, the
\href{https://github.com/henrik-andersen/FE-SEM/blob/master/simulation-code.R}{simulated
dataset} generates multiple \emph{indicators} of the independent and
dependent variables that all measure the intended variable imprecisely.
Returning to our panel data, we have three indicators of each the
independent and dependent variable, per timepoint:\\
\begin{align}
\tilde{x}_{kt} & = x_{t} + \upsilon_{kt}, \\
\tilde{y}_{kt} & = y_{t} + \nu_{kt}
\end{align} where \(k = 1, 2, 3\) and \(t = 1, ..., T\). This is like
repeatedly presenting a respondent with a multi-item scale designed to
measure things like stress, depression, xenophobia, etc. over the course
of a panel study. To create the observed indicators, a random amount of
measurement error (ranging from
\(\{\sigma^{2}_{\upsilon_{k}}, \sigma^{2}_{\nu_{k}}\}\in \{1.0, 1.1, 1.2, 1.3, 1.4, 1.5\}\))
was added to the true variables, again see
\href{https://github.com/henrik-andersen/FE-SEM/blob/master/simulation-code.R}{the
simulation code}.

Let us first focus on the issue of imprecise measurements of the
independent variable of interest and run the same FE-SEM model above,
but this time we will use one of the measurement error sullied
indicators, here \(\tilde{x}_{1t}\), instead of the true independent
variable, \(x_{t}\). As for the naming conventions in the \texttt{R}
code, \texttt{x11} stands for the first indicator (\(k = 1\)) at the
first point in time (\(t = 1\)), whereas for example \texttt{x35} stands
for the third indicator (\(k = 3\)) at the fifth point in time
(\(t = 5\)).

\singlespacing

\begin{Shaded}
\begin{Highlighting}[]
\NormalTok{fe\_sem2 \textless{}{-}}\StringTok{ \textquotesingle{}}
\StringTok{\# Define individual effects variable }
\StringTok{a =\textasciitilde{} 1*y1 + 1*y2 + 1*y3 + 1*y4 + 1*y5}
\StringTok{\# Regressions, constrain coefficient to be equal over time}
\StringTok{\# Now the imprecisely measured indicator tilde\{x\}\_kt}
\StringTok{\# instead of the true variable x\_t}
\StringTok{y1 \textasciitilde{} b*x11 }
\StringTok{y2 \textasciitilde{} b*x12 }
\StringTok{y3 \textasciitilde{} b*x13}
\StringTok{y4 \textasciitilde{} b*x14}
\StringTok{y5 \textasciitilde{} b*x15}
\StringTok{\# Allow unrestricted correlation between eta and covariates}
\StringTok{a \textasciitilde{}\textasciitilde{} x11 + x12 + x13 + x14 + x15}
\StringTok{x11 \textasciitilde{}\textasciitilde{} x12 + x13 + x14 + x15}
\StringTok{x12 \textasciitilde{}\textasciitilde{} x13 + x14 + x15}
\StringTok{x13 \textasciitilde{}\textasciitilde{} x14 + x15}
\StringTok{x14 \textasciitilde{}\textasciitilde{} x15}
\StringTok{\# Constrain residual variances to be equal over time}
\StringTok{y1 \textasciitilde{}\textasciitilde{} e*y1}
\StringTok{y2 \textasciitilde{}\textasciitilde{} e*y2}
\StringTok{y3 \textasciitilde{}\textasciitilde{} e*y3}
\StringTok{y4 \textasciitilde{}\textasciitilde{} e*y4}
\StringTok{y5 \textasciitilde{}\textasciitilde{} e*y5}
\StringTok{\textquotesingle{}}
\NormalTok{fe\_sem2.fit \textless{}{-}}\StringTok{ }\KeywordTok{sem}\NormalTok{( }\DataTypeTok{model =}\NormalTok{ fe\_sem2, }
                    \DataTypeTok{data =}\NormalTok{ dfw, }
                    \DataTypeTok{estimator =} \StringTok{"ML"}\NormalTok{)}
\end{Highlighting}
\end{Shaded}

\doublespacing

Now, for the sake of brevity, let us look just at the estimated
coefficients for \(\beta\).

\singlespacing

\begin{Shaded}
\begin{Highlighting}[]
\KeywordTok{summary}\NormalTok{( fe\_sem2.fit)}
\end{Highlighting}
\end{Shaded}

\begin{verbatim}
...
##                    Estimate  Std.Err  z-value  P(>|z|)
##   y1 ~                                                
##     x11        (b)    0.091    0.009    9.611    0.000
##   y2 ~                                                
##     x12        (b)    0.091    0.009    9.611    0.000
##   y3 ~                                                
##     x13        (b)    0.091    0.009    9.611    0.000
##   y4 ~                                                
##     x14        (b)    0.091    0.009    9.611    0.000
##   y5 ~                                                
##     x15        (b)    0.091    0.009    9.611    0.000
## 
...
\end{verbatim}

\doublespacing

Obviously, the estimated coefficient \(\hat{\beta} =\) 0.091 is
substantially smaller than the true population coefficient of
\(\beta = 0.3\). And the discrepancy is not just due to sampling error.
In fact, we can derive the bias we are observing here.

For a simple bivariate regression model, it is straightforward to
quantify the bias due to measurement error. It will be \begin{align}
\mathop{\mathrm{\mathrm{Cov}}}(y, \tilde{x}) & = \mathop{\mathrm{\mathbb{E}}}[y \tilde{x}] \\
 & = \mathop{\mathrm{\mathbb{E}}}[(\beta\tilde{x} + \varepsilon)\tilde{x}] \\
 & = \mathop{\mathrm{\mathbb{E}}}[\beta\tilde{x}^{2} + \varepsilon \tilde{x}] \\
 & = \beta \mathop{\mathrm{\mathrm{Var}}}(\tilde{x}) \\
\hat{\beta} & = \frac{\mathop{\mathrm{\mathrm{Cov}}}(y, \tilde{x})}{\mathop{\mathrm{\mathrm{Var}}}(\tilde{x})} \\
 & = \frac{\mathop{\mathrm{\mathbb{E}}}[(\beta x + \varepsilon)(x + \upsilon)]}{\mathop{\mathrm{\mathbb{E}}}[(x + \upsilon)^{2}]} \\
 & = \frac{\mathop{\mathrm{\mathbb{E}}}[\beta x^{2} + \beta x \upsilon + \varepsilon x + \varepsilon \upsilon]}{\mathop{\mathrm{\mathbb{E}}}[x^{2} + 2 x \upsilon + \upsilon^{2}]} \\
 & = \beta \frac{\mathop{\mathrm{\mathrm{Var}}}(x)}{\mathop{\mathrm{\mathrm{Var}}}(x) + \mathop{\mathrm{\mathrm{Var}}}(\upsilon)}.
\end{align} which results if we assume that
\(\mathop{\mathrm{\mathbb{E}}}[x \upsilon] = 0\),
\(\mathop{\mathrm{\mathbb{E}}}[x \varepsilon] = 0\),
\(\mathop{\mathrm{\mathbb{E}}}[\tilde{x} \varepsilon] = 0\) and
\(\mathop{\mathrm{\mathbb{E}}}[\varepsilon \upsilon] = 0\)
\citep{Wooldridge2009}. However, the model we are interested is not a
bivariate model, so what was the point of showing the this? For one, it
points out that the bias will always move the estimated coefficient
closer to 0, since
\(\mathop{\mathrm{\mathrm{Var}}}(x) \le \mathop{\mathrm{\mathrm{Var}}}(x) + \mathop{\mathrm{\mathrm{Var}}}(\upsilon)\).
This means positive effects will be biased downwards and negative
effects biased upwards, always towards zero. This is why it is referred
to as \emph{attenuation bias}. Second, it will help to familiarize
ourselves with this equation to better understand the one for the
multivariate case.

Indeed, the magnitude of the bias in a multivariate model is somewhat
more complex to derive, but it will be \begin{align}
\hat{\beta} & = \beta \frac{\mathop{\mathrm{\mathrm{Var}}}(\theta)}{\mathop{\mathrm{\mathrm{Var}}}(\theta) + \mathop{\mathrm{\mathrm{Var}}}(\upsilon)}
\end{align} where \(\theta\) is just the residual of a regression in
which the underlying theoretical variable is regressed on all other
covariates. In this case, we need to regress \(x_{t}\) on \(\alpha_{1}\)
and \(\alpha_{2}\) for:
\(x_{t} = \tau + \gamma_{1}\alpha_{1} + \gamma_{2}\alpha_{2} + \theta_{t}\)
where \(\tau\) is the intercept, and \(\gamma_{1}\), \(\gamma_{2}\) are
the regression coefficients and \(\theta_{t}\) is the residual
\citep[p.~318--320]{Wooldridge2009}.

Normally it is not possible to reconstruct the bias since in cases where
we have to rely on indicators, we would not have observed the underlying
theoretical variable. Furthermore, in the case of a fixed-effects model,
the covariates are the unobserved time-invariant characteristics.
However, because we are working with simulated data, we have everything
we need. Going back to the results above, we can get the residuals of
\(x_{t}\) by either running a regression and saving the residuals, or we
could skip a step and get them directly using the `residual maker'
matrix \citep{Ruettenauer2020} which is
\(\bm{M} = \bm{I} - \bm{A}(\bm{A}^{\intercal}\bm{A})^{-1}\bm{A}^{\intercal}\)
and
\(\bm{A} = \begin{pmatrix}\bm{\iota}_{n} & \bm{\alpha_{1}} & \bm{\alpha_{2}}\end{pmatrix}\)
is the \(n \times 3\) matrix of the covariates (plus a constant).

\singlespacing

\begin{Shaded}
\begin{Highlighting}[]
\CommentTok{\# Make the n x n identity matrix}
\NormalTok{Id \textless{}{-}}\StringTok{ }\KeywordTok{diag}\NormalTok{( n)}

\CommentTok{\# n x 2 matrix of covariates a1 and a2}
\NormalTok{A \textless{}{-}}\StringTok{ }\KeywordTok{matrix}\NormalTok{( }\KeywordTok{c}\NormalTok{( }\KeywordTok{rep}\NormalTok{( }\DecValTok{1}\NormalTok{, n), dfw}\OperatorTok{$}\NormalTok{a1, dfw}\OperatorTok{$}\NormalTok{a2), }
             \DataTypeTok{nrow =}\NormalTok{ n, }\DataTypeTok{ncol =} \DecValTok{3}\NormalTok{)}

\CommentTok{\# The residual maker matrix M = I {-} A(A\textquotesingle{}A)\^{}{-}1 A\textquotesingle{}}
\NormalTok{M \textless{}{-}}\StringTok{ }\NormalTok{Id }\OperatorTok{{-}}\StringTok{ }\NormalTok{A }\OperatorTok{\%*\%}\StringTok{ }\KeywordTok{solve}\NormalTok{( }\KeywordTok{t}\NormalTok{( A) }\OperatorTok{\%*\%}\StringTok{ }\NormalTok{A) }\OperatorTok{\%*\%}\StringTok{ }\KeywordTok{t}\NormalTok{( A)}

\CommentTok{\# Save the residuals, t for \textquotesingle{}theta\textquotesingle{}}
\NormalTok{t \textless{}{-}}\StringTok{ }\NormalTok{M }\OperatorTok{\%*\%}\StringTok{ }\NormalTok{dfw}\OperatorTok{$}\NormalTok{x1}

\CommentTok{\# Re{-}run the FE model from above with the \textquotesingle{}true\textquotesingle{} }
\CommentTok{\# independent variable for the correct estimate for beta }
\NormalTok{fe\_sem.fit \textless{}{-}}\StringTok{ }\KeywordTok{sem}\NormalTok{( }\DataTypeTok{model =}\NormalTok{ fe\_sem, }\DataTypeTok{data =}\NormalTok{ dfw, }\DataTypeTok{estimator =} \StringTok{"ML"}\NormalTok{)}

\CommentTok{\# The equation for the biased beta }
\KeywordTok{lavInspect}\NormalTok{( fe\_sem.fit, }\StringTok{"list"}\NormalTok{)[ }\DecValTok{6}\NormalTok{, }\DecValTok{14}\NormalTok{]}\OperatorTok{*}
\StringTok{  }\NormalTok{(( }\KeywordTok{var}\NormalTok{( t))}\OperatorTok{/}\NormalTok{( }\KeywordTok{var}\NormalTok{( t) }\OperatorTok{+}\StringTok{ }\KeywordTok{var}\NormalTok{( dfw}\OperatorTok{$}\NormalTok{x11 }\OperatorTok{{-}}\StringTok{ }\NormalTok{dfw}\OperatorTok{$}\NormalTok{x1)))}
\end{Highlighting}
\end{Shaded}

\begin{verbatim}
##           [,1]
## [1,] 0.1027325
\end{verbatim}

\doublespacing

\singlespacing

\doublespacing

From this we can see that the biased estimate above of \(\hat{\beta} =\)
0.091 roughly comes from
\(\beta \frac{\mathop{\mathrm{\mathrm{Var}}}(\theta_{t})}{\mathop{\mathrm{\mathrm{Var}}}(\theta_{t}) + \mathop{\mathrm{\mathrm{Var}}}(\upsilon_{t})} = 0.294 \frac{1.054}{3.018}\)
= 0.103; `roughly' because the equation here is the population equation.
Due to sampling error, the estimates will tend vary slightly.

\hypertarget{using-latent-variables-to-deal-with-measurement-error}{%
\subsubsection{Using latent variables to deal with measurement
error}\label{using-latent-variables-to-deal-with-measurement-error}}

The way we deal with measurement error in SEM is surprisingly similar to
the logic of fixed-effects regression. Namely, if we have multiple
cross-sectional observations of the underlying construct of interest,
then we can define a latent variable that represents the common variance
across those multiple variables. Contrast this with the use of
longitudinal repeated measures to isolate the common variance across
time.

So, if we do in fact have multiple cross-sectional indicators for the
underlying variables of interest, then we can partition them into an
explained and unexplained portion:\\
\begin{align}
x_{kt} & = \lambda_{kt}^{x}\xi_{t} + \delta_{kt}, \\
y_{kt} & = \lambda_{kt}^{y}\eta_{t} + \varepsilon_{kt},
\end{align} where \(x_{kt}\) and \(y_{kt}\) are the \(k^{th}\)
indicators, \(\xi_{t}\) and \(\eta_{t}\) are latent factors representing
the common variance across the cross-sectional repeated measures, and
\(\delta_{kt}\) and \(\varepsilon_{kt}\) are the unexplained portions of
\(x_{t}\) and \(y_{t}\), respectively. The latent factors are linked to
the observed indicators through the factor loadings \(\lambda_{kt}\).

Thus, our FE regression equation changes from
\(y_{t} = \beta x_{t} + \alpha + \varepsilon_{t}\) to: \begin{align}
\eta_{t} & = \beta \xi_{t} + \alpha + \zeta_{t}
\end{align} where \(\zeta_{t}\) represents the disturbance, in other
words the residual of the latent dependent variable \(\eta_{t}\). First,
however, let us double-check that measurement error in the dependent
variable only increases the error variance (thus also increasing
standard errors and reducing \(R^{2}\)), but does not systematically
bias the coefficients of interest. The next model uses the indicators of
\(x\) and specifies latent variables (\(\xi_{t}\), \texttt{xi} in the
code) to represent the valid cross-sectional variance. The dependent
variable in the model is one of the impercisely measured indicators of
\(y\).

\singlespacing

\begin{Shaded}
\begin{Highlighting}[]
\NormalTok{fe\_sem3 \textless{}{-}}\StringTok{ \textquotesingle{}}
\StringTok{\# Define individual effects variable }
\StringTok{a =\textasciitilde{} 1*y11 + 1*y12 + 1*y13 + 1*y14 + 1*y15}
\StringTok{\# Measurement model for independent variables, xi }
\StringTok{xi1 =\textasciitilde{} 1*x11 + x21 + x31 }
\StringTok{xi2 =\textasciitilde{} 1*x12 + x22 + x32}
\StringTok{xi3 =\textasciitilde{} 1*x13 + x23 + x33}
\StringTok{xi4 =\textasciitilde{} 1*x14 + x24 + x34}
\StringTok{xi5 =\textasciitilde{} 1*x15 + x25 + x35}
\StringTok{\# Regressions, constrain coefficient to be equal over time}
\StringTok{y11 \textasciitilde{} b*xi1}
\StringTok{y12 \textasciitilde{} b*xi2 }
\StringTok{y13 \textasciitilde{} b*xi3}
\StringTok{y14 \textasciitilde{} b*xi4}
\StringTok{y15 \textasciitilde{} b*xi5}
\StringTok{\# Allow unrestricted correlation between eta and covariates}
\StringTok{a \textasciitilde{}\textasciitilde{} xi1 + xi2 + xi3 + xi4 + xi5}
\StringTok{xi1 \textasciitilde{}\textasciitilde{} xi2 + xi3 + xi4 + xi5}
\StringTok{xi2 \textasciitilde{}\textasciitilde{} xi3 + xi4 + xi5}
\StringTok{xi3 \textasciitilde{}\textasciitilde{} xi4 + xi5}
\StringTok{xi4 \textasciitilde{}\textasciitilde{} xi5}
\StringTok{\# Constrain residual variances to be equal over time}
\StringTok{y11 \textasciitilde{}\textasciitilde{} e*y11}
\StringTok{y12 \textasciitilde{}\textasciitilde{} e*y12}
\StringTok{y13 \textasciitilde{}\textasciitilde{} e*y13}
\StringTok{y14 \textasciitilde{}\textasciitilde{} e*y14}
\StringTok{y15 \textasciitilde{}\textasciitilde{} e*y15}
\StringTok{\textquotesingle{}}
\NormalTok{fe\_sem3.fit \textless{}{-}}\StringTok{ }\KeywordTok{sem}\NormalTok{( }\DataTypeTok{model =}\NormalTok{ fe\_sem3, }
                    \DataTypeTok{data =}\NormalTok{ dfw, }
                    \DataTypeTok{estimator =} \StringTok{"ML"}\NormalTok{)}
\end{Highlighting}
\end{Shaded}

\doublespacing

\singlespacing

\begin{Shaded}
\begin{Highlighting}[]
\KeywordTok{summary}\NormalTok{( fe\_sem3.fit)}
\end{Highlighting}
\end{Shaded}

\begin{verbatim}
...
## Regressions:
##                    Estimate  Std.Err  z-value  P(>|z|)
##   y11 ~                                               
##     xi1        (b)    0.299    0.029   10.302    0.000
##   y12 ~                                               
##     xi2        (b)    0.299    0.029   10.302    0.000
##   y13 ~                                               
##     xi3        (b)    0.299    0.029   10.302    0.000
##   y14 ~                                               
##     xi4        (b)    0.299    0.029   10.302    0.000
##   y15 ~                                               
##     xi5        (b)    0.299    0.029   10.302    0.000
...
\end{verbatim}

\doublespacing

The estimated coefficient here in model \texttt{fe\_sem3.fit} is
\(\hat{\beta}_{y_{1t},\xi_{t}} =\) 0.299 which is very close to the
estimated coefficient in the first, correctly specified model
\texttt{fe\_sem.fit}, where \(\hat{\beta}_{y_{t},x_{t}} =\) 0.294.
Notice, however, that the standard error of the estimate is
substantially larger, with 0.029 in \texttt{fe\_sem3.fit} vs.~0.016 in
\texttt{fe\_sem.fit} in which \(y\) was measured without error. The
explained variance (\(R^{2}\)) in the dependent variable was also much
higher in the first model:

\singlespacing

\begin{Shaded}
\begin{Highlighting}[]
\KeywordTok{lavInspect}\NormalTok{( fe\_sem.fit, }\StringTok{"r2"}\NormalTok{)[ }\DecValTok{1}\OperatorTok{:}\DecValTok{5}\NormalTok{]}
\end{Highlighting}
\end{Shaded}

\begin{verbatim}
##        y1        y2        y3        y4        y5 
## 0.5893174 0.5928913 0.5894869 0.5853815 0.5845328
\end{verbatim}

\doublespacing

compared to the current model:

\singlespacing

\begin{Shaded}
\begin{Highlighting}[]
\KeywordTok{lavInspect}\NormalTok{( fe\_sem3.fit, }\StringTok{"r2"}\NormalTok{)[ }\DecValTok{1}\OperatorTok{:}\DecValTok{5}\NormalTok{]}
\end{Highlighting}
\end{Shaded}

\begin{verbatim}
##       y11       y12       y13       y14       y15 
## 0.3901187 0.3914316 0.3828739 0.3850362 0.3703136
\end{verbatim}

\doublespacing

Finally, to see the benefits of removing measurement error from the
dependent variable in terms of standard errors and \(R^{2}\) statistics,
we can specify a model with latent variables representing the valid
cross-sectional variance in \(y\) (\texttt{n} for \(\eta\) in the code).

\singlespacing

\begin{Shaded}
\begin{Highlighting}[]
\NormalTok{fe\_sem4 \textless{}{-}}\StringTok{ \textquotesingle{}}
\StringTok{\# Measurement model for dependent variable, n for eta}
\StringTok{n1 =\textasciitilde{} 1*y11 + y21 + y31}
\StringTok{n2 =\textasciitilde{} 1*y12 + y22 + y32}
\StringTok{n3 =\textasciitilde{} 1*y13 + y23 + y33}
\StringTok{n4 =\textasciitilde{} 1*y14 + y24 + y34}
\StringTok{n5 =\textasciitilde{} 1*y15 + y25 + y35}
\StringTok{\# Define individual effects variable }
\StringTok{a =\textasciitilde{} 1*n1 + 1*n2 + 1*n3 + 1*n4 + 1*n5}
\StringTok{\# Measurement model for independent variables, xi }
\StringTok{xi1 =\textasciitilde{} 1*x11 + x21 + x31 }
\StringTok{xi2 =\textasciitilde{} 1*x12 + x22 + x32}
\StringTok{xi3 =\textasciitilde{} 1*x13 + x23 + x33}
\StringTok{xi4 =\textasciitilde{} 1*x14 + x24 + x34}
\StringTok{xi5 =\textasciitilde{} 1*x15 + x25 + x35}
\StringTok{\# Regressions, constrain coefficient to be equal over time}
\StringTok{n1 \textasciitilde{} b*xi1}
\StringTok{n2 \textasciitilde{} b*xi2 }
\StringTok{n3 \textasciitilde{} b*xi3}
\StringTok{n4 \textasciitilde{} b*xi4}
\StringTok{n5 \textasciitilde{} b*xi5}
\StringTok{\# Allow unrestricted correlation between eta and covariates}
\StringTok{a \textasciitilde{}\textasciitilde{} xi1 + xi2 + xi3 + xi4 + xi5}
\StringTok{xi1 \textasciitilde{}\textasciitilde{} xi2 + xi3 + xi4 + xi5}
\StringTok{xi2 \textasciitilde{}\textasciitilde{} xi3 + xi4 + xi5}
\StringTok{xi3 \textasciitilde{}\textasciitilde{} xi4 + xi5}
\StringTok{xi4 \textasciitilde{}\textasciitilde{} xi5}
\StringTok{\# Constrain residual variances to be equal over time}
\StringTok{n1 \textasciitilde{}\textasciitilde{} e*n1}
\StringTok{n2 \textasciitilde{}\textasciitilde{} e*n2}
\StringTok{n3 \textasciitilde{}\textasciitilde{} e*n3}
\StringTok{n4 \textasciitilde{}\textasciitilde{} e*n4}
\StringTok{n5 \textasciitilde{}\textasciitilde{} e*n5}
\StringTok{\textquotesingle{}}
\NormalTok{fe\_sem4.fit \textless{}{-}}\StringTok{ }\KeywordTok{sem}\NormalTok{( }\DataTypeTok{model =}\NormalTok{ fe\_sem4, }
                    \DataTypeTok{data =}\NormalTok{ dfw, }
                    \DataTypeTok{estimator =} \StringTok{"ML"}\NormalTok{)}
\end{Highlighting}
\end{Shaded}

\doublespacing

\singlespacing

\begin{Shaded}
\begin{Highlighting}[]
\KeywordTok{summary}\NormalTok{( fe\_sem4.fit)}
\end{Highlighting}
\end{Shaded}

\begin{verbatim}
...
## Regressions:
##                    Estimate  Std.Err  z-value  P(>|z|)
##   n1 ~                                                
##     xi1        (b)    0.264    0.023   11.515    0.000
##   n2 ~                                                
##     xi2        (b)    0.264    0.023   11.515    0.000
##   n3 ~                                                
##     xi3        (b)    0.264    0.023   11.515    0.000
##   n4 ~                                                
##     xi4        (b)    0.264    0.023   11.515    0.000
##   n5 ~                                                
##     xi5        (b)    0.264    0.023   11.515    0.000
...
\end{verbatim}

\doublespacing

Here, the effect \(\hat{\beta}_{\eta_{t},\xi_{t}}\) is somewhat further
off of the true effect of 0.3 than the preceding models. This will
depend on how the latent variables are estimated, which themselves will
depend on the underlying correlations between the indicators. Again, if
the main goal of the model is to avoid bias, it may be advisable to just
leave the manifest dependent variable as it is, and worry about
measurement error in the independent variables.

\hypertarget{time-invariant-predictors}{%
\subsection{Time-invariant predictors}\label{time-invariant-predictors}}

What if we do not just want to just control for the effects of all
time-invariant variables, but investigate some of them in detail? Many
time-invariant variables, like sex, birth cohort, nationality,
education, etc. can be interesting on their own. And typically, many of
these variables are readily available in a given dataset. The
traditional OLS-based FE model does not allow for this, as it wipes out
the effect of \emph{all} time-invariant variables, whether observed or
not.

In SEM, we can easily specify a type of \emph{hybrid} FE/RE model
\citep{Bollen2010} that allows us to control for time-invariant
unobserved heterogeneity while also investigating the effects of
specific observed time-invariant predictors.\footnote{These types of
  models have become well known outside of SEM as well, see for example
  \citet{Allison2011}; \citet{Schunck2013}; \citet{Bell2018}.}

In the next example, we continue with the most complex model we have
specified so far, \texttt{fe\_sem4.fit} in which measurement error in
both the independent and dependent variables is accounted for using
latent variables. Now, we would like as well to specifically investigate
the effect of \(\alpha_{2}\) on the dependent variable. The equation for
this model changes to:
\(\eta_{t} = \beta \xi_{t} + \alpha + \gamma \alpha_{2} + \zeta_{t}\).

\singlespacing

\begin{Shaded}
\begin{Highlighting}[]
\NormalTok{fe\_sem5 \textless{}{-}}\StringTok{ \textquotesingle{}}
\StringTok{\# Measurement model for dependent variable, n for eta}
\StringTok{n1 =\textasciitilde{} 1*y11 + y21 + y31}
\StringTok{n2 =\textasciitilde{} 1*y12 + y22 + y32}
\StringTok{n3 =\textasciitilde{} 1*y13 + y23 + y33}
\StringTok{n4 =\textasciitilde{} 1*y14 + y24 + y34}
\StringTok{n5 =\textasciitilde{} 1*y15 + y25 + y35}
\StringTok{\# Define individual effects variable }
\StringTok{a =\textasciitilde{} 1*n1 + 1*n2 + 1*n3 + 1*n4 + 1*n5}
\StringTok{\# Measurement model for independent variables, xi }
\StringTok{xi1 =\textasciitilde{} 1*x11 + x21 + x31 }
\StringTok{xi2 =\textasciitilde{} 1*x12 + x22 + x32}
\StringTok{xi3 =\textasciitilde{} 1*x13 + x23 + x33}
\StringTok{xi4 =\textasciitilde{} 1*x14 + x24 + x34}
\StringTok{xi5 =\textasciitilde{} 1*x15 + x25 + x35}
\StringTok{\# Regressions, constrain coefficient to be equal over time}
\StringTok{n1 \textasciitilde{} b*xi1 + g*a2}
\StringTok{n2 \textasciitilde{} b*xi2 + g*a2}
\StringTok{n3 \textasciitilde{} b*xi3 + g*a2}
\StringTok{n4 \textasciitilde{} b*xi4 + g*a2}
\StringTok{n5 \textasciitilde{} b*xi5 + g*a2}
\StringTok{\# Allow unrestricted correlation between eta and covariates}
\StringTok{a \textasciitilde{}\textasciitilde{} xi1 + xi2 + xi3 + xi4 + xi5 + 0*a2}
\StringTok{a2 \textasciitilde{}\textasciitilde{} xi1 + xi2 + xi3 + xi4 + xi5}
\StringTok{xi1 \textasciitilde{}\textasciitilde{} xi2 + xi3 + xi4 + xi5}
\StringTok{xi2 \textasciitilde{}\textasciitilde{} xi3 + xi4 + xi5}
\StringTok{xi3 \textasciitilde{}\textasciitilde{} xi4 + xi5}
\StringTok{xi4 \textasciitilde{}\textasciitilde{} xi5}
\StringTok{\# Constrain residual variances to be equal over time}
\StringTok{n1 \textasciitilde{}\textasciitilde{} e*n1}
\StringTok{n2 \textasciitilde{}\textasciitilde{} e*n2}
\StringTok{n3 \textasciitilde{}\textasciitilde{} e*n3}
\StringTok{n4 \textasciitilde{}\textasciitilde{} e*n4}
\StringTok{n5 \textasciitilde{}\textasciitilde{} e*n5}
\StringTok{\textquotesingle{}}
\NormalTok{fe\_sem5.fit \textless{}{-}}\StringTok{ }\KeywordTok{sem}\NormalTok{( }\DataTypeTok{model =}\NormalTok{ fe\_sem5, }
                    \DataTypeTok{data =}\NormalTok{ dfw, }
                    \DataTypeTok{estimator =} \StringTok{"ML"}\NormalTok{)}
\end{Highlighting}
\end{Shaded}

\doublespacing

Keep in mind, based on the DGP, the true parameters are \(\beta = 0.3\)
and \(\gamma = 0.45\).

\singlespacing

\begin{Shaded}
\begin{Highlighting}[]
\KeywordTok{summary}\NormalTok{( fe\_sem5.fit)}
\end{Highlighting}
\end{Shaded}

\begin{verbatim}
...
## Regressions:
##                    Estimate  Std.Err  z-value  P(>|z|)
##   n1 ~                                                
##     xi1        (b)    0.265    0.023   11.515    0.000
##     a2         (g)    0.490    0.033   14.999    0.000
##   n2 ~                                                
##     xi2        (b)    0.265    0.023   11.515    0.000
##     a2         (g)    0.490    0.033   14.999    0.000
##   n3 ~                                                
##     xi3        (b)    0.265    0.023   11.515    0.000
##     a2         (g)    0.490    0.033   14.999    0.000
##   n4 ~                                                
##     xi4        (b)    0.265    0.023   11.515    0.000
##     a2         (g)    0.490    0.033   14.999    0.000
##   n5 ~                                                
##     xi5        (b)    0.265    0.023   11.515    0.000
##     a2         (g)    0.490    0.033   14.999    0.000
...
\end{verbatim}

\doublespacing

From this we can see that such a hybrid model is does a good job of
estimating the coefficients of interest, with \(\hat{\beta} =\) 0.265
(0.023) and \(\hat{\gamma} =\) 0.49 (0.033).

It is important, however, to realize that the unbiasedness of
\(\hat{\gamma}\) in this model is dependent on the assumption that
\(\mathop{\mathrm{\mathbb{E}}}[\zeta | \bm{\xi_{t}}, \alpha_{2}] = 0\).
In other words, the idiosyncratic error is mean independent of
\(\bm{\xi_{t}} = (\xi_{1}, \xi_{2}, ..., \xi_{T})\) as well as
\(\alpha_{2}\). The first part is easier to accept because we are
controlling for all potential time-invariant confounders that could
induce a relationship between the independent variable and the error.
The unbiasedness of \(\hat{\gamma}\), on the other hand rests on the
assumption that the time-invariant predictor is independent of the
error. If \(\alpha_{2}\) represented the respondent's intelligence and
\(\eta_{t}\), the dependent variable, represented the respondent's
income, for example, then \(\hat{\gamma}\) would be biased if both were
dependent on a third time-invariant variable, say level of schooling, if
it is not controlled for. For this reason, we need to treat the
regression on a time-invariant predictor like any other regular
multivariate regression model and look to include all plausible
potential confounders as controls in the model, or turn to other
methods, e.g., instrumental variables.

\bibliographystyle{tfcad}
\bibliography{references2.bib}




\end{document}
